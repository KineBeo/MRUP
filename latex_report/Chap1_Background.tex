\clearpage
\phantomsection

\setcounter{chapter}{0}
\chapter{Kiến thức nền tảng}

\noindent Chương này giới thiệu các kiến thức nền tảng về window functions trong SQL và SQLite, hệ quản trị cơ sở dữ liệu được sử dụng trong nghiên cứu.

\section{Window Functions trong SQL}

\subsection{Khái niệm cơ bản}

\noindent Window functions là tính năng được giới thiệu trong SQL:2003, cho phép thực hiện phép tính trên tập hợp các hàng liên quan đến hàng hiện tại mà vẫn giữ nguyên số lượng hàng. Khác với GROUP BY làm giảm số hàng, window functions thêm cột tính toán cho mỗi hàng dựa trên "cửa sổ" các hàng liên quan.

{Cú pháp chung của Window Function:}
\vspace{-0.6em}
\begin{center}
\begin{lstlisting} 
function_name(expression) OVER (
  [PARTITION BY column_list]
  [ORDER BY column_list]
  [frame_clause]
)
\end{lstlisting} 
\end{center}
\vspace{-0.6em}
Cú pháp tổng quát bao gồm tên hàm và mệnh đề OVER chứa window specification với ba thành phần: PARTITION BY chia dữ liệu thành các phân vùng độc lập, ORDER BY xác định thứ tự các hàng, và frame clause xác định tập hợp hàng được sử dụng. Với bảng dữ liệu employees như sau, ta có thể so sánh kết quả của truy vấn GROUP BY và window function:

\begin{table}[H]
	\centering
    \small
    \begin{tabular}{|l|l|r|}
\hline
\textbf{name} & \textbf{dept} & \textbf{salary} \\ \hline
Alice & IT & 5000 \\ \hline
Bob & IT & 6000 \\ \hline
Carol & HR & 4000 \\ \hline
Dave & HR & 5000 \\ \hline
Emma & IT & 7000 \\ \hline
\end{tabular}
    \caption{Bảng dữ liệu employees sử dụng cho ví dụ}
    \label{tab:employees_example}
\end{table}

\begin{table}[H]
\centering
\small
\begin{tabular}{|p{6.5cm}|p{7.5cm}|}
\hline
\textbf{GROUP BY (Aggregate)} & \textbf{Window Function} \\ \hline

\texttt{SELECT dept, AVG(salary)} &
\texttt{SELECT name, dept, salary,} \\

\texttt{FROM employees} &
\texttt{AVG(salary) OVER (PARTITION BY dept)} \\

\texttt{GROUP BY dept;} &
\texttt{FROM employees;} \\ \hline

\textbf{Kết quả:} &
\textbf{Kết quả:} \\ \hline

\begin{tabular}{@{}ll@{}}
IT & 6000.0 \\
HR & 4500.0
\end{tabular}
&
\begin{tabular}{@{}llll@{}}
Carol & HR & 4000 & 4500.0 \\
Dave  & HR & 5000 & 4500.0 \\
Alice & IT & 5000 & 6000.0 \\
Bob   & IT & 6000 & 6000.0 \\
Emma  & IT & 7000 & 6000.0
\end{tabular}
\\ \hline

\end{tabular}
\caption{So sánh GROUP BY và Window Function}
\label{tab:groupby_vs_window}
\end{table}


\subsection{Các thành phần chính}

\subsubsection{PARTITION BY}

\noindent PARTITION BY chia tập kết quả thành các partition độc lập. Window function tính toán riêng cho mỗi partition, không có tác động qua lại giữa các partition. Đây là tính chất partition locality, nền tảng cho quan hệ metamorphic MRUP.

\begin{table}[H]
	\centering
    \small
    \begin{tabular}{|l|l|r|r|}
\hline
\textbf{name} & \textbf{dept} & \textbf{salary} & \textbf{avg\_dept} \\ \hline
Alice & IT & 5000 & 5500 \\
Bob & IT & 6000 & 5500 \\
Carol & HR & 4000 & 4500 \\
Dave & HR & 5000 & 4500 \\ \hline
\end{tabular}
    \caption{Ví dụ PARTITION BY dept: mỗi partition (IT, HR) có trung bình riêng}
    \label{tab:partition_by_example}
\end{table}

\subsubsection{ORDER BY}

\noindent ORDER BY xác định thứ tự các hàng trong partition, ảnh hưởng trực tiếp đến các hàm ranking như ROW\_NUMBER, RANK và frame specification.

\begin{table}[H]
    \centering
    \small
    \begin{tabular}{|l|r|r|r|}
\hline
\textbf{name} & \textbf{salary} & \textbf{ROW\_NUMBER} & \textbf{RANK} \\ \hline
Bob & 6000 & 1 & 1 \\
Alice & 5000 & 2 & 2 \\
Dave & 5000 & 3 & 2 \\
Carol & 4000 & 4 & 4 \\ \hline
\end{tabular}
    \caption{Ví dụ ORDER BY salary DESC: ROW\_NUMBER liên tục, RANK có gap}
    \label{tab:order_by_example}
\end{table}

\subsubsection{Frame Specification}

\noindent Frame specification xác định tập hợp hàng trong partition để tính toán. SQL hỗ trợ ROWS frame (đếm theo hàng vật lý) và RANGE frame (đếm theo giá trị logic). Frame boundaries bao gồm UNBOUNDED PRECEDING, N PRECEDING, CURRENT ROW, N FOLLOWING, và UNBOUNDED FOLLOWING.

\begin{table}[H]
    \centering
    \small
    \begin{tabular}{|l|r|r|r|}
\hline
\textbf{day} & \textbf{sales} & \textbf{3-day MA} & \textbf{cumulative} \\ \hline
2024-01-01 & 100 & 100 & 100 \\
2024-01-02 & 150 & 125 & 250 \\
2024-01-03 & 200 & 150 & 450 \\
2024-01-04 & 180 & 177 & 630 \\ \hline
\end{tabular}
    \caption{ROWS 2 PRECEDING (3-day moving average) vs UNBOUNDED PRECEDING (cumulative)}
    \label{tab:frame_example}
\end{table}

\subsection{Phân loại Window Functions}

\noindent Window functions có thể được chia thành ba nhóm chính: \textit{ranking functions}, \textit{aggregate window functions} và \textit{value functions}.

\textit{Ranking functions} (ROW\_NUMBER, RANK, DENSE\_RANK) dùng để xác định thứ hạng của các dòng trong mỗi partition. ROW\_NUMBER đánh số liên tục, RANK cho phép xuất hiện khoảng trống khi có giá trị trùng nhau, trong khi DENSE\_RANK vẫn đảm bảo thứ hạng liên tiếp. [Bảng~\ref{tab:ranking_functions}].

\textit{Aggregate window functions} (SUM, AVG, COUNT, MIN, MAX) thực hiện phép tổng hợp trên một cửa sổ dữ liệu. Khác với GROUP BY, các hàm này không làm giảm số dòng của kết quả và có thể tính toán trên toàn bộ partition hoặc một frame cụ thể. [Bảng~\ref{tab:aggregate_functions}].

\textit{Value functions} (LAG, LEAD, FIRST\_VALUE, LAST\_VALUE) cho phép truy cập giá trị của các dòng khác trong cùng partition, hỗ trợ phân tích dữ liệu theo chiều thời gian hoặc so sánh giữa các dòng mà không cần sử dụng self-join. [Bảng~\ref{tab:value_aggregate_functions}]y.

\begin{table}[H]
    \centering
    \small
    \begin{tabular}{|l|r|r|r|r|}
\hline
\textbf{name} & \textbf{salary} & \textbf{ROW\_NUMBER} & \textbf{RANK} & \textbf{DENSE\_RANK} \\ \hline
Bob & 6000 & 1 & 1 & 1 \\
Alice & 5000 & 2 & 2 & 2 \\
Dave & 5000 & 3 & 2 & 2 \\
Carol & 4000 & 4 & 4 & 3 \\ \hline
\end{tabular}
    \caption{Ranking Functions: ROW\_NUMBER liên tục, RANK có gap, DENSE\_RANK không gap}
    \label{tab:ranking_functions}
\end{table}

\begin{table}[H]
	\centering
    \small
    \begin{tabular}{|l|l|r|r|}
\hline
\textbf{name} & \textbf{dept} & \textbf{salary} & \textbf{sum\_dept} \\ \hline
Alice & IT & 5000 & 18000 \\
Bob & IT & 6000 & 18000 \\
Carol & HR & 4000 & 9000 \\
Dave & HR & 5000 & 9000 \\ \hline
\end{tabular}
    \caption{Ví dụ Aggregate Window Function: SUM(salary) OVER (PARTITION BY dept)}
    \label{tab:aggregate_functions}
\end{table}

\begin{table}[H]
	\centering
    \small
    \begin{tabular}{|l|r|r|r|r|}
\hline
\textbf{month} & \textbf{revenue} & \textbf{prev\_month} & \textbf{change} & \textbf{cumulative} \\ \hline
Jan & 1000 & NULL & NULL & 1000 \\
Feb & 1200 & 1000 & +200 & 2200 \\
Mar & 1100 & 1200 & -100 & 3300 \\
Apr & 1300 & 1100 & +200 & 4600 \\ \hline
\end{tabular}
    \caption{Value function LAG() và Aggregate function SUM() với UNBOUNDED PRECEDING}
    \label{tab:value_aggregate_functions}
\end{table}


\section{SQLite và Window Functions}

\subsection{Tổng quan về SQLite}

\noindent SQLite là DBMS nhúng (embedded), mã nguồn mở, hoạt động serverless với database lưu trong một file. SQLite có kích thước nhỏ gọn (~600KB), không cần cấu hình, chạy đa nền tảng, và có độ tin cậy cao với test coverage >100\%. SQLite được sử dụng trong hàng tỷ thiết bị từ smartphone đến trình duyệt web.

Window functions được giới thiệu trong SQLite 3.25.0 (9/2018). Tuy nhiên, SQLite có một số hạn chế so với SQL chuẩn.

\begin{table}[H]
    \centering
    \small
    \begin{tabular}{|l|p{5cm}|p{6cm}|}
\hline
\textbf{Hạn chế} & \textbf{Mô tả} & \textbf{Ràng buộc MRUP} \\ \hline
RANGE + Multiple ORDER BY & Không hỗ trợ RANGE frame với ORDER BY nhiều cột & C4: RANGE frame → ORDER BY 1 cột \\ \hline
Frame + Ranking & Không cho phép frame với ROW\_NUMBER, RANK, DENSE\_RANK & C3: Ranking functions → không có frame \\ \hline
Dynamic Typing & Một cột có thể chứa nhiều kiểu dữ liệu & Cần type-aware comparison \\ \hline
\end{tabular}
    \caption{Các hạn chế của SQLite với Window Functions}
    \label{tab:sqlite_limitations}
\end{table}

SQLite xử lý window functions qua pipeline: parsing (xây dựng AST) → validation (kiểm tra ràng buộc) → planning (tạo query plan) → execution (thực thi trên từng partition) → materialization (trả kết quả). Lỗi logic có thể xảy ra ở mọi bước, đặc biệt frame boundary calculation với NULL, partition sorting, type coercion, và optimization bugs.

\section{Thách thức trong kiểm thử Window Functions}

\subsection{Không gian trạng thái lớn}

\noindent Window functions tạo ra không gian trạng thái rất lớn: 15+ loại hàm × 2 loại frame (ROWS, RANGE) × 5 loại boundaries × 4 loại exclusions ≈ 3000+ configurations, chưa kể ORDER BY, PARTITION BY và tương tác với WHERE, JOIN, GROUP BY, subqueries.

\begin{table}[H]
	\centering
    \small
    \begin{tabular}{|l|r|p{7cm}|}
\hline
\textbf{Thành phần} & \textbf{Số lượng} & \textbf{Ví dụ} \\ \hline
Window Functions & 15+ & ROW\_NUMBER, RANK, SUM, AVG, LAG, LEAD... \\ \hline
Frame Types & 2 & ROWS, RANGE \\ \hline
Frame Boundaries & 5 & UNBOUNDED PRECEDING, N PRECEDING, CURRENT ROW... \\ \hline
Frame Exclusions & 4 & EXCLUDE CURRENT ROW, EXCLUDE GROUP... \\ \hline
\textbf{Tổ hợp} & \textbf{3000+} & Chưa kể ORDER BY, PARTITION BY, WHERE, JOIN... \\ \hline
\end{tabular}
    \caption{Không gian trạng thái của Window Functions}
    \label{tab:state_space}
\end{table}

Tính không xác định (non-determinism) xảy ra khi ORDER BY không đủ xác định thứ tự duy nhất, hoặc khi sử dụng RANDOM(), CURRENT\_TIMESTAMP. NULL handling không rõ ràng (thiếu NULLS FIRST/LAST) cũng gây kết quả không nhất quán.
